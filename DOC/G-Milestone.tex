\documentclass{article}

\title{Paladim - Milestone hand-in}
\author{}

\begin{document}
\maketitle
\newpage
\section{Parser implementation}
We've implemented the parser in mosmlyac using the grammar given in the GroupProject.pdf
as a starting point. To make the grammar suitable for LR(1)-parsing we made some
transformations of the grammar:
\begin{enumerate}
  \item We've made \texttt{if-then-else}-statements right-associative (simply by using
        \texttt{\%right}) in mosmlyac. This means that an \texttt{if} will bind to the
        closest \texttt{else}. This is the only transformation that actually change the
        language though it was necessary as the given grammar is ambiguous.
  \item All nonterminals of the form $X \rightarrow \alpha | X \alpha$ has been transformed
        to $X \rightarrow \alpha | \alpha X$. This has been done as SML-lists are most
        effectively constructed by using \texttt{::}.
  \item In the production $Exp \rightarrow Exp \; OP \; Exp$, $OP$ has been replaced with
        the actual operator-terminals. This was done to unhide operator precedence for
        mosmlyac.
        %This was done to let mosmlyac see what level of operator precedence there is
        %during the parsing of expressions of the form 
        %This was done to make mosmlyac generate a state for each precedence level of the
        %operators. Otherwise, it wouldn't be able to know what kind of operator (if any)
        %the preceding expression contained.$
\end{enumerate}
\section{Testing}


\section{Ideas for other tasks}

\end{document}
